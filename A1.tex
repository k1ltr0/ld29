\documentclass[a4paper, 10pt]{article}
\usepackage[letterpaper,left=2cm,right=2cm,top=2cm,bottom=2cm]{geometry}
\usepackage{amsmath}
\usepackage{graphicx}
\DeclareGraphicsExtensions{.bmp,.png,.jpg,.pdf}
\usepackage[spanish]{babel}
\usepackage{courier}
\usepackage{textcomp}
\usepackage[utf8]{inputenc}
\usepackage{amssymb}

\begin{document}
\begin{quote}
\begin{minipage}{1.5cm}
\end{minipage}  
 \begin{minipage}{6cm}
 \vspace{0.5cm}
 
\end{minipage}
\begin{center}

{\huge Trayectorias del juego}\\
\vspace{0.3 cm}

\hrulefill
\end{center}
 
\textbf{Campo de fluidos}

Se resuelve la ecuacion diferencial ordinaria de segundo orden

$$m \ddot r = mg - \beta \dot r $$

Hacemos el cambio de variables $\dot r = v$ e intentamos con el ansatz $v=Ae^{\lambda t}$ para la homogenea

$$m \dot v + \beta v = mg $$

$$ m \lambda A e^{\lambda t}+ \beta e^{\lambda t}=0 $$

$$ \rightarrow \lambda= - \frac{\beta }{m} $$


si $v= cte=k$

$$ k = mg/\beta$$

$$ r(t)= Ae^{- \frac{\beta }{m} t}+ \frac{mg}{\beta} $$
Por lo tanto solo afecta la velocidad, decae exponencialmente y llega a la velocidad terminal 

\textbf{Super gravedad}


Las ecuaciones a resolver son


$$m \ddot r = r\dot \theta^2 -mg $$
$$ m(r\ddot \theta + 2 \dot r \dot \theta)=0 $$

No hay solucion analitica, por lo tanto, aproximamos la solucion como una elipse donde uno de los focos es el centro de masa entre ambos cuerpos (cañon y mosntruo). La solucion a la posicion es 

$$x(t)= (a sin(t), b cos(t))$$

debemos encontrar entonces las constantes a y b (semi eje mayor y menor) de acuerdo a la posicion del cañon.

Primero debemos calcular el tiempo en que demora en recorrer la proyeccion  en la linea que une el centro del cañon con el mosntruo (eje central)

$$ asin(t)= Lcos(\phi) \quad \rightarrow \quad t_0= arcsin( \frac{L}{a}cos(\phi)) $$

donde L es el largo del cañon y $\phi$ es el angulo de incidencia.
 
Luego la velocidad de la bala es

$$ v(t)= (acos(t),-bsin(t)) $$
entones por condiciones iniciales en  el instante $t_0$ debe ser la velocidad inicial

$$  v_0 cos(\phi)= acos( arcsin( \frac{L}{a}cos(\phi)))= a \sqrt{1-  (\frac{L}{a})^2} $$

$$ v_0 sin(\phi) = -bsin( arcsin( \frac{L}{a}cos(\phi)))=-b \frac{L}{a}cos(\phi)) $$

Resolvemos el sistema de ecuaciones

$$ a= \sqrt{v_0^2 cos(\phi)^2 +L^2} \qquad b=- \frac{v_0 sin(\phi)}{L}\sqrt{v_0^2cos(\phi)^2 +L^2} $$

finalmente la ecuaciones son:

$$x(t)= ( \sqrt{v_0^2 cos(\phi)^2 +L^2} sin(t), - \frac{v_0 sin(\phi)}{L}\sqrt{v_0^2cos(\phi)^2 +L^2}cos(t))$$

notar que al derivar tenemos que la constate que acompaña a t esta implicita y vale 1, solo sale la dimesion que es inversa al tiempo

ademas $$ lvl = \lambda v_i $$
\textbf{Electromagnetismo}

Un campo magnetico constante con un elecrico constante perpendicular producen una trayectoria de una cicloide.


la ecuacion es

$$x(t) = a(t-sin(t)), a(1-cos(t)) $$


\textbf{Inercia- velocidad angulares distintas dependiendo del radio, masa y tentaculos}

Usando conservacion de momento angular

$$\omega_1^2 I_1 = I_2 \omega_2^2 $$

donde $\omega_i$ son las velocidades angulares en diferentes momentos (1 es la inicial) y $I$ es el tensor de inercia


$$I_1= MR^2$$
M masa central y R es el radio inicial (200 pixeles de diametro el monstruo  creo, por lo tanto R=100 pixeles)

$$I_2 = \sum_{i=1}^5 m_i (R_1 + (i-1)\delta R)^2 $$ 
ya que cada anillo tiene una diferencia de $\delta R$ y ademas son solo 5

pero ademas depende la masa de cada capa, luego si cada capa se divide en $N_i$ (pueden ser distintos) y cada capa tiene $n_i$ divisiones consideradas(pueden ser distintos)
$$I_2= \sum_{i=1}^5  m_o \frac{n_i}{N_i}( R_1 + (i-1)\delta R )^2 = m_o(R_1^2 +\delta R^2)\sum_{i=1}^5\frac{n_i}{ N_i} + 2 \delta Rm_o \sum_{i=1}^5 \frac{n_i}{N_i}(i-1)  $$

Luego la nueva velocidad es


$$\omega_2 = \omega_1 \sqrt{\frac{m_oR^2}{m_o(R_1^2 +\delta R^2)\sum_{i=1}^5\frac{n_i}{ N_i} + 2 \delta Rm_o \sum_{i=1}^5 \frac{n_i}{N_i}(i-1)}}$$

\textbf{velocidad de salida tencaculo}
$$v= \sqrt{F_o\cdot L / m} $$

$F_o$ es la fuerza que le da, puede ser un valor random cada ves
$L$ largo del tentaculo (onda cuantas capas ocupa, de 1 a 5)
$m$ masa del tentaculo (que depende del numero en que se divide cada capa, y del largo)








\end{quote}
\end{document}